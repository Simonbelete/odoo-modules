\addcontentsline{toc}{chapter}{Abstract}
\begin{center}
\section*{ABSTRACT}
\end{center}
\begin{normalsize}
\doublespacing
In today's fast growing of internet users and fast-changing business environment in Ethiopia, it's extremely important to be able to respond to client needs in the most effective and timely manner.Online Shopping is a lifestyle e-commerce web application, which retails various fashion and lifestyle products.The primary goal of an e-commerce site is to sell goods online. This project with developing an e-commerce website and mobile application for Online Product Sale. It provides the user with a catalog of different product available for purchase in the store, viewing various products available enables registered users to purchase desired products process payment by using Cash on Delivery(Pay Later) or Mobile Banking. In order to facilitate online purchase a shopping cart is provided to the user and delivery which is fulfilled by the express system.

Nowadays, many different kinds of delivery companies in Ethiopia transport their own kinds of parcels and offer their own services, which have caused a lot waste of resource. In addition , the volume of parcels in all cities that need to be delivered has been grown dramatically. To cope with these problems, Guya-Express System in the country which can offer service to all kinds of customers in the city including manufactures, department stores, restaurants, individual people and so forth will be designed. This system use combining computer network technology, wireless communication and cloud computing. With this system. the whole package delivery process including classification of packages, vehicle scheduling, path planning, transportation monitoring can be intellectualized as well as managed automatically, and the use of both material resources and manpower resources can be reduced accordingly.

This document will discuss each of the underlying technologies to create and implement and e-commerce and express under the name Guya E-commerce and Guya Express respectively and for architectural implementation we will be using Microservices Architecture.
\end{normalsize}
